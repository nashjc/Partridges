\documentclass[a4paper]{article}

\usepackage[pages=all, color=black, position={current page.south}, placement=bottom, scale=1, opacity=1, vshift=5mm]{background}
\SetBgContents{
	\tt This work is shared under a \href{https://creativecommons.org/licenses/by-sa/4.0/}{CC BY-SA 4.0 license} unless otherwise noted
}      % copyright

\usepackage[margin=1in]{geometry} % full-width

% AMS Packages
\usepackage{amsmath}
\usepackage{amsthm}
\usepackage{amssymb}

% Unicode
\usepackage[utf8]{inputenc}
\usepackage{hyperref}
\hypersetup{
	unicode,
%	colorlinks,
%	breaklinks,
%	urlcolor=cyan, 
%	linkcolor=blue, 
	pdfauthor={Author One, Author Two, Author Three},
	pdftitle={A simple article template},
	pdfsubject={A simple article template},
	pdfkeywords={article, template, simple},
	pdfproducer={LaTeX},
	pdfcreator={pdflatex}
}

% Vietnamese
%\usepackage{vntex}

% Natbib
\usepackage[sort&compress,numbers,square]{natbib}
\bibliographystyle{mplainnat}

% Theorem, Lemma, etc
\theoremstyle{plain}
\newtheorem{theorem}{Theorem}
\newtheorem{corollary}[theorem]{Corollary}
\newtheorem{lemma}[theorem]{Lemma}
\newtheorem{claim}{Claim}[theorem]
\newtheorem{axiom}[theorem]{Axiom}
\newtheorem{conjecture}[theorem]{Conjecture}
\newtheorem{fact}[theorem]{Fact}
\newtheorem{hypothesis}[theorem]{Hypothesis}
\newtheorem{assumption}[theorem]{Assumption}
\newtheorem{proposition}[theorem]{Proposition}
\newtheorem{criterion}[theorem]{Criterion}
\theoremstyle{definition}
\newtheorem{definition}[theorem]{Definition}
\newtheorem{example}[theorem]{Example}
\newtheorem{remark}[theorem]{Remark}
\newtheorem{problem}[theorem]{Problem}
\newtheorem{principle}[theorem]{Principle}

\usepackage{graphicx, color}
\graphicspath{{fig/}}

%\usepackage[linesnumbered,ruled,vlined,commentsnumbered]{algorithm2e} % use algorithm2e for typesetting algorithms
%% \usepackage{algorithm, algpseudocode} % use algorithm and algorithmicx for typesetting algorithms
%% \usepackage{mathrsfs} % for \mathscr command  -- Functionality similar to this package 
%% has recently(everypage) been implemented in LaTeX. This package is now in(everypage) legacy status.

\usepackage{lipsum}

% Author info
\title{A simple article template}
\author{Author One$^1$\thanks{Author One was partially supported by Grant XXX} \and Author Two$^2$ \and Author Three$^1$}

\date{
	$^1$Organization 1 \\ \texttt{\{auth1, auth3\}@org1.edu}\\%
	$^2$Organization 2 \\ \texttt{auth3@inst2.edu}\\[2ex]%
%	\today
}

\begin{document}
	\maketitle
	
	\begin{abstract}
		\lipsum[1]
		
		\noindent\textbf{Keywords:} article, template, simple
	\end{abstract}

	\tableofcontents
	
	\section{Introduction}
	\label{sec:intro}
	
	\lipsum[2]
	
	\subsection{Preliminaries}
	\label{sec:pre}
	
	\lipsum[3]
	
	\subsection{Previous Results}
	\label{sec:prev-results}
	
	Null graphs are discussed in \cite{HararyR74}
	The concept of ``internally stable set'' was used in \cite{Berge57, Berge58}.
	
	\begin{theorem}
		\label{thrm:1}
		\lipsum[4]
	\end{theorem}
	\begin{proof}
		content...
	\end{proof}

	\begin{corollary}
	\label{cor:1}
	
	\lipsum[5]
	\end{corollary}

	Unordered List (taken from Overleaf)
	\begin{itemize}
		\item The individual entries are indicated with a black dot, a so-called bullet.
		\item The text in the entries may be of any length.
	\end{itemize}

	Ordered List (taken from Overleaf)
	\begin{enumerate}
		\item The labels consists of sequential numbers.
		\item The numbers starts at 1 with every call to the enumerate environment.
	\end{enumerate}

	\begin{table}[ht]
		\centering
		\begin{tabular}{|c|c|}
			\hline
			\textbf{Odd} & \textbf{Even} \\
			\hline\hline
			One & Two \\
			\hline
			Three & Four \\
			\hline
		\end{tabular}
		\caption{This is a table}
		\label{tbl:1}
	\end{table}

	Table~\ref*{tbl:1} is an example of a table.
title: "Partridges in Pear Trees -- Rmarkdown"
author: "John Nash and Prashanth Velayudhan"
date: "2025-12-08"

\section*{Motivation}

The preparation of scientific documents with mathematical content has always given
their authors and printers difficulty in rendering the material to paper or screen.
This article is part of an exercise to illustrate how different systems for preparing
such documents compare in capability and ease of use.

The text preparation systems to be compared are LaTeX (??ref), Typst (??ref) and
Rmarkdown (??ref). We note that bibliographic support is also of interest to users.

\section{A useful example}

The Christmas counting song "The Twelve Days of Christmas" suggest the singer 
receives one gift -- we shall unabashedly simply declare that the Partridge in 
a Pear Tree is a single gift -- on the first of twelve days, then that gift plus
two more on the second, and so forth. For our needs, we want to have a general
formula for the total number of gifts received after $n$ days. While the commonly
known song uses 12 days, there are variants with other values of $n$. The Faroe
Islands use the inflationary value $n = 15$. In our exposition, we will rely on
the Wikipedia reference ??ref as our authority.

Thus we seek a formula for $T(n)$, the total number of presents received after
the $n$'th day.

\section*{Single day number of presents}

On a single day, the number of presents $S(n)$ is clearly the sum of an arithmetic
progression (??ref)

?? How to label equations on the same line??

$$ S(k) = \sum _{i = 1} ^k {i} $$
Clearly we now want to compute

$$ T(n) = \sum _{j = 1} ^n S(k) $$
The formula for $S(k)$ is well known, and derived by noting that writing the sequence forwards and then backwards illustrates that twice the sum is $k * (k+1)$, so we have

$$  S(k) = k (k + 1) / 2  $$
We can use this so that we find

$$  2 T(n) =  \sum _{k=1} ^n  {( k^2 + k )}  $$

We therefore need 

$$ Q(n) = \sum _{k = 1} ^n {k^2}$$

To give our provisional expression as

$$  T(n) = ( Q(n) + S(n) ) / 2 $$

$Q(n)$ is a well-known summation, often proved by mathematical induction,

$$  Q(n) = n (2 n + 1) (n + 1) / 6 $$
Thus we want (?? how to do multiline equation?)

$$       T(n) = ( n (2 n + 1 )(n+1) / 6 + n(n+1)/ 2) / 2 
            = ( 2 n^3 + 3 n^2 + n  + 3 n^2 + 3 n) / 12
            = ( 2 n^3 + 6 n^2 + 4 n)/12
            = ( n^3 + 3 n^2 + 2 n)/6  
$$

\bibliography{references.bib}
	
\end{document}
